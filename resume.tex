\documentclass[11pt,a4paper,roman,english]{moderncv}        % possible options include font size ('10pt', '11pt' and '12pt'), paper size ('a4paper', 'letterpaper', 'a5paper', 'legalpaper', 'executivepaper' and 'landscape') and font family ('sans' and 'roman')
\moderncvstyle{banking}                             % style options are 'casual' (default), 'classic', 'oldstyle' and 'banking'
\moderncvcolor{black}                               % color options 'blue' (default), 'orange', 'green', 'red', 'purple', 'grey' and 'black'
%\nopagenumbers{}                                  % uncomment to suppress automatic page numbering for CVs longer than one page
\usepackage[utf8]{inputenc}                       % if you are not using xelatex ou lualatex, replace by the encoding you are using
\usepackage[scale=0.75,a4paper]{geometry}
\usepackage{multicol}
\usepackage{babel}
%\quote{An eccentric programmer who likes math and computer science.}
%----------------------------------------------------------------------------------
%            personal data
%----------------------------------------------------------------------------------
\firstname{Thomas}
\familyname{Gebert}
                              % optional, remove/comment the line if not wanted
\address{693 Thomas S Boyland St.}{Brooklyn, NY}{United States}         % optional, remove/comment the line if not wanted; the "country" arguments can be omitted or provided empty

                          % optional, remove/comment the line if not wanted
\phone{214-451-7333}                           % optional, remove/comment the line if not wanted
                         % optional, remove/comment the line if not wanted
\email{thomas@gebert.app}                               % optional, remove/comment the line if not wanted
% \homepage{www.gebert.sexy}                         % optional, remove/comment the line if not wanted
                 % optional, remove/comment the line if not wanted
% \photo[64pt][0.4pt]{picture}                       % optional, uncomment the line if wanted; '64pt' is the height the picture must be resized to, 0.4pt is the thickness of the frame around it (put it to 0pt for no frame) and 'picture' is the name of the picture file
                              % optional, remove/comment the line if not wanted
%
\begin{document}
%-----       resume       ---------------------------------------------------------


\makecvtitle
\section{Computer skills}
\begin{multicols}{4}
	\begin{itemize}
		\item[] \textbf{Python}
		\item[] \textbf{SQL}
		\item[] \textbf{Clojure}
		\item[] \textbf{Haskell}
		\item[] \textbf{Erlang/OTP}
		\item[] Java
		\item[] ZeroMQ
		\item[] C
		\item[] F\#
		\item[] Go
		\item[] \textbf{Kakfa}
		\item[] JavaScript
		\item[] \textbf{Apache Spark}
		\item[] Node.JS
		\item[] Git
		\item[] Scala
		\item[] \textbf{FFMpeg}
        \item[] TLA+
	\end{itemize}
\end{multicols}
\section{Education}
\cventry{}{B.S. Computer Science}{Western Governor's University}{Salt Lake City, UT}{}{} 
\cventry{}{Ph.D. Computer Science}{University of York}{York, United Kingdom}{Formal Methods}{Currently Enrolled} 
\section{Experience}
\cventry{August 2022 -- June 2023}{Adjunct Lecturer}{CUNY City Tech}{New York, New York}{}{\begin{itemize}
	\item Taught introductory programming to computer engineering students.  \begin{itemize}
		\item Introductory and advanced Python and Java. 
	\end{itemize}
    \item Created and graded homework assignments and tests.
    \item Coded interactive learning modules in JavaScript and Python for students to practice.
    \item Worked with professors designing cyber-physical systems to help formalize robotics applications.
\end{itemize}}

\cventry{August 2021 -- August 2022}{Staff Software Engineer}{Walmart Global Tech}{New York, New York}{}{\begin{itemize}
	\item Built NLU pipelines for the Chatbot on walmart.com utilizing the Microsoft Bot Framework. \begin{itemize}
		\item Utilized the Microsoft Bot Framework, F\# and C\#. 
	\end{itemize}
        \item Coordinated the release of multi-cluster deployment of chatbot. 
        \item Utilized Microsoft Azure, Linux, and Python to build infrastructure. 
\end{itemize}}
\cventry{September 2018 -- February 2021}{Senior Software Engineer}{Apple Inc.}{New York, New York}{}{\begin{itemize}
	\item Designed and built a telemetry and analytics system for finding potential bottlenecks in the cache indexes utilizing Java, Clojure, Kafka, Apache Spark, and Tableau. 
	\item Fixed issues and bugs in the iTunes server backend. 
        \item Built a Kafka-based buffering service to reduce load on indexing and caching for iTunes. 
	\item Maintained and expanded rule engine for iTunes, utilizing Objective-C and C++. 
    \item Deployed Java code onto Amazon Web Services and Linux.
\end{itemize}}
\cventry{July 2016 -- August 2018}{Senior Software Engineer}{Walmart Global Tech (Jet.com)}{Hoboken, New Jersey}{}{\begin{itemize}
    \item Wrote Microservices in F\#.
    \item Utilized the Microsoft Azure stack.
    \item Used Apache Kafka to send data between services.
    \item Rebuilt the transactional email system to scale to Jet.com size. 
    \item Taught the F\# language during code bootcamps. 
\end{itemize}}
\cventry{September 2015 -- June 2016}{Software Engineer}{Tone Mobile}{New York, New York}{}{\begin{itemize}
		\item Write and maintain Erlang backends.
		\item Create modules for Ejabberd.
		\item Integrate chat server with Node.js backend.
	\end{itemize}}
\cventry{March 2015 -- September 2015}{Research Scientist}{New York University}{New York, New York}{}{\begin{itemize}
		\item Debug Angular.JS frontends.
		\item Debug Scala backends. 
		\item Help write Haskell backends. 
		\item Programmatically use FFMpeg for video transcoding. 
	\end{itemize}}
\cventry{May 2014 -- February 2015}{Application Developer}{Sq1}{Dallas, Texas}{}{}
\cventry{June 2013 -- April 2014}{Software Engineer}{Senico, LLC}{Dallas, Texas}{}{}
\cventry{January 2013 -- June 2013}{Software Engineer}{Propulsion Labs}{Dallas, Texas}{}{ }
\cventry{January 2012 -- December 2012}{Web Developer}{Amerinational Management Services}{Orlando, Florida}{}{}
\cventry{May 2011 -- August 2011}{Software Engineering Intern}{Lockheed Martin}{Orlando, Florida}{}{}

\section{Public Speaking}

\cvitem{Lambda Days 2023 -- Why Design Your Own Levels When Your Computer Can Do it?}{Presented an introduction to doing WebGL with ClojureScript, in addition to showing a few simple procedural level generation algorithms for games}

\cvitem{Lambda Days 2022 -- Predicting and Preventing Chaos with Formal Methods in TLA+}{Presented in Krakow, Poland, an introduction to formal methods via a brief description and demonstration of the TLA+ specification language. }

\cvitem{Lambda Days 2020 -- Distributed Hash Tables, Video, and Fun!}{Presented in Krakow, Poland, the same talk as stated above.}

\cvitem{Clojure Conj 2019 -- Distributed Hash Tables, Video, and Fun!}{Presented in Durham, North Carolina, a demonstration of a project involving a video sharing system using distributed hash tables and to farm out video transcoding.}

\section{Side Projects}
\cvitem{FSharp.Csv}{A reflection-based CSV serializer, written in F\#, designed to handle large, multi-gigabyte CSV files, while providing a simple, functional interface, and remaining relatively fast. }
\cvitem{Vertigo.Json}{A reflection-based JSON serializer and deserializer, designed to be used with F\#, with an emphasis on being easy-to-use, fast, and null-safe. }
\begin{center}
For more, please visit gitlab.com/tombert
\end{center}
%\section{References}
%\begin{cvcolumns}
 % \cvcolumn{Bobby Gammill}{Colleague: 407-913-9973}
 % \cvcolumn{Darren Green}{Coworker: 214-302-7807}
  %\cvcolumn{Will Nielsen}{Manager: 972-983-6228}
  
%\end{cvcolumns}
%\begin{cvcolumns}
 % \cvcolumn{Owen Ambrose}{Coworker: 337-335-8520}
 %   \cvcolumn{Michael Lage}{Coworker: 214-918-4569}
 %   \cvcolumn{Ron Chin}{Coworker: 347-765-3846}
%\end{cvcolumns}
\clearpage
\end{document}
